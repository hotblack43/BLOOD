\documentclass[]{article}
\usepackage{lmodern}
\usepackage{amssymb,amsmath}
\usepackage{ifxetex,ifluatex}
\usepackage{fixltx2e} % provides \textsubscript
\ifnum 0\ifxetex 1\fi\ifluatex 1\fi=0 % if pdftex
  \usepackage[T1]{fontenc}
  \usepackage[utf8]{inputenc}
\else % if luatex or xelatex
  \ifxetex
    \usepackage{mathspec}
  \else
    \usepackage{fontspec}
  \fi
  \defaultfontfeatures{Ligatures=TeX,Scale=MatchLowercase}
\fi
% use upquote if available, for straight quotes in verbatim environments
\IfFileExists{upquote.sty}{\usepackage{upquote}}{}
% use microtype if available
\IfFileExists{microtype.sty}{%
\usepackage{microtype}
\UseMicrotypeSet[protrusion]{basicmath} % disable protrusion for tt fonts
}{}
\usepackage[margin=1in]{geometry}
\usepackage{hyperref}
\hypersetup{unicode=true,
            pdftitle={Testing Bllod Pressure},
            pdfborder={0 0 0},
            breaklinks=true}
\urlstyle{same}  % don't use monospace font for urls
\usepackage{color}
\usepackage{fancyvrb}
\newcommand{\VerbBar}{|}
\newcommand{\VERB}{\Verb[commandchars=\\\{\}]}
\DefineVerbatimEnvironment{Highlighting}{Verbatim}{commandchars=\\\{\}}
% Add ',fontsize=\small' for more characters per line
\usepackage{framed}
\definecolor{shadecolor}{RGB}{248,248,248}
\newenvironment{Shaded}{\begin{snugshade}}{\end{snugshade}}
\newcommand{\AlertTok}[1]{\textcolor[rgb]{0.94,0.16,0.16}{#1}}
\newcommand{\AnnotationTok}[1]{\textcolor[rgb]{0.56,0.35,0.01}{\textbf{\textit{#1}}}}
\newcommand{\AttributeTok}[1]{\textcolor[rgb]{0.77,0.63,0.00}{#1}}
\newcommand{\BaseNTok}[1]{\textcolor[rgb]{0.00,0.00,0.81}{#1}}
\newcommand{\BuiltInTok}[1]{#1}
\newcommand{\CharTok}[1]{\textcolor[rgb]{0.31,0.60,0.02}{#1}}
\newcommand{\CommentTok}[1]{\textcolor[rgb]{0.56,0.35,0.01}{\textit{#1}}}
\newcommand{\CommentVarTok}[1]{\textcolor[rgb]{0.56,0.35,0.01}{\textbf{\textit{#1}}}}
\newcommand{\ConstantTok}[1]{\textcolor[rgb]{0.00,0.00,0.00}{#1}}
\newcommand{\ControlFlowTok}[1]{\textcolor[rgb]{0.13,0.29,0.53}{\textbf{#1}}}
\newcommand{\DataTypeTok}[1]{\textcolor[rgb]{0.13,0.29,0.53}{#1}}
\newcommand{\DecValTok}[1]{\textcolor[rgb]{0.00,0.00,0.81}{#1}}
\newcommand{\DocumentationTok}[1]{\textcolor[rgb]{0.56,0.35,0.01}{\textbf{\textit{#1}}}}
\newcommand{\ErrorTok}[1]{\textcolor[rgb]{0.64,0.00,0.00}{\textbf{#1}}}
\newcommand{\ExtensionTok}[1]{#1}
\newcommand{\FloatTok}[1]{\textcolor[rgb]{0.00,0.00,0.81}{#1}}
\newcommand{\FunctionTok}[1]{\textcolor[rgb]{0.00,0.00,0.00}{#1}}
\newcommand{\ImportTok}[1]{#1}
\newcommand{\InformationTok}[1]{\textcolor[rgb]{0.56,0.35,0.01}{\textbf{\textit{#1}}}}
\newcommand{\KeywordTok}[1]{\textcolor[rgb]{0.13,0.29,0.53}{\textbf{#1}}}
\newcommand{\NormalTok}[1]{#1}
\newcommand{\OperatorTok}[1]{\textcolor[rgb]{0.81,0.36,0.00}{\textbf{#1}}}
\newcommand{\OtherTok}[1]{\textcolor[rgb]{0.56,0.35,0.01}{#1}}
\newcommand{\PreprocessorTok}[1]{\textcolor[rgb]{0.56,0.35,0.01}{\textit{#1}}}
\newcommand{\RegionMarkerTok}[1]{#1}
\newcommand{\SpecialCharTok}[1]{\textcolor[rgb]{0.00,0.00,0.00}{#1}}
\newcommand{\SpecialStringTok}[1]{\textcolor[rgb]{0.31,0.60,0.02}{#1}}
\newcommand{\StringTok}[1]{\textcolor[rgb]{0.31,0.60,0.02}{#1}}
\newcommand{\VariableTok}[1]{\textcolor[rgb]{0.00,0.00,0.00}{#1}}
\newcommand{\VerbatimStringTok}[1]{\textcolor[rgb]{0.31,0.60,0.02}{#1}}
\newcommand{\WarningTok}[1]{\textcolor[rgb]{0.56,0.35,0.01}{\textbf{\textit{#1}}}}
\usepackage{graphicx,grffile}
\makeatletter
\def\maxwidth{\ifdim\Gin@nat@width>\linewidth\linewidth\else\Gin@nat@width\fi}
\def\maxheight{\ifdim\Gin@nat@height>\textheight\textheight\else\Gin@nat@height\fi}
\makeatother
% Scale images if necessary, so that they will not overflow the page
% margins by default, and it is still possible to overwrite the defaults
% using explicit options in \includegraphics[width, height, ...]{}
\setkeys{Gin}{width=\maxwidth,height=\maxheight,keepaspectratio}
\IfFileExists{parskip.sty}{%
\usepackage{parskip}
}{% else
\setlength{\parindent}{0pt}
\setlength{\parskip}{6pt plus 2pt minus 1pt}
}
\setlength{\emergencystretch}{3em}  % prevent overfull lines
\providecommand{\tightlist}{%
  \setlength{\itemsep}{0pt}\setlength{\parskip}{0pt}}
\setcounter{secnumdepth}{0}
% Redefines (sub)paragraphs to behave more like sections
\ifx\paragraph\undefined\else
\let\oldparagraph\paragraph
\renewcommand{\paragraph}[1]{\oldparagraph{#1}\mbox{}}
\fi
\ifx\subparagraph\undefined\else
\let\oldsubparagraph\subparagraph
\renewcommand{\subparagraph}[1]{\oldsubparagraph{#1}\mbox{}}
\fi

%%% Use protect on footnotes to avoid problems with footnotes in titles
\let\rmarkdownfootnote\footnote%
\def\footnote{\protect\rmarkdownfootnote}

%%% Change title format to be more compact
\usepackage{titling}

% Create subtitle command for use in maketitle
\providecommand{\subtitle}[1]{
  \posttitle{
    \begin{center}\large#1\end{center}
    }
}

\setlength{\droptitle}{-2em}

  \title{Testing Bllod Pressure}
    \pretitle{\vspace{\droptitle}\centering\huge}
  \posttitle{\par}
    \author{}
    \preauthor{}\postauthor{}
    \date{}
    \predate{}\postdate{}
  

\begin{document}
\maketitle

\hypertarget{setup}{%
\subsection{Setup}\label{setup}}

\begin{Shaded}
\begin{Highlighting}[]
\KeywordTok{library}\NormalTok{(lubridate)}
\end{Highlighting}
\end{Shaded}

\begin{verbatim}
## 
## Attaching package: 'lubridate'
\end{verbatim}

\begin{verbatim}
## The following object is masked from 'package:base':
## 
##     date
\end{verbatim}

\begin{Shaded}
\begin{Highlighting}[]
\KeywordTok{library}\NormalTok{(fBasics)}
\end{Highlighting}
\end{Shaded}

\begin{verbatim}
## Loading required package: timeDate
\end{verbatim}

\begin{verbatim}
## Loading required package: timeSeries
\end{verbatim}

\hypertarget{get-my-bp-readings}{%
\subsection{Get my BP readings}\label{get-my-bp-readings}}

\begin{Shaded}
\begin{Highlighting}[]
\NormalTok{data <-}\StringTok{ }\KeywordTok{read.csv}\NormalTok{(}\StringTok{"bp3.csv"}\NormalTok{,}\DataTypeTok{header=}\NormalTok{T,}\DataTypeTok{sep=}\StringTok{""}\NormalTok{,}\DataTypeTok{fileEncoding =} \StringTok{"UCS-2LE"}\NormalTok{)}
\KeywordTok{colnames}\NormalTok{(data) <-}\StringTok{ }\KeywordTok{c}\NormalTok{(}\StringTok{"Year"}\NormalTok{,}\StringTok{"Month"}\NormalTok{,}\StringTok{"Day"}\NormalTok{,}\StringTok{"Hour"}\NormalTok{,}\StringTok{"Minute"}\NormalTok{,}\StringTok{"Sys"}\NormalTok{,}\StringTok{"Dia"}\NormalTok{,}\StringTok{"Pulse"}\NormalTok{)}
\NormalTok{POSIXtime =}\StringTok{ }\KeywordTok{ISOdatetime}\NormalTok{(}\DataTypeTok{year=}\NormalTok{data}\OperatorTok{$}\NormalTok{Year, }\DataTypeTok{month=}\NormalTok{data}\OperatorTok{$}\NormalTok{Month, }\DataTypeTok{day=}\NormalTok{data}\OperatorTok{$}\NormalTok{Day, }\DataTypeTok{hour =}\NormalTok{ data}\OperatorTok{$}\NormalTok{Hour, }\DataTypeTok{min =}\NormalTok{ data}\OperatorTok{$}\NormalTok{Minute, }\DataTypeTok{sec =} \DecValTok{0}\NormalTok{, }\DataTypeTok{tz =} \StringTok{"UTC"}\NormalTok{) }\CommentTok{# on the hour}
\NormalTok{jd <-}\StringTok{ }\KeywordTok{julian}\NormalTok{(POSIXtime)}
\NormalTok{local_df <-}\StringTok{ }\KeywordTok{cbind}\NormalTok{(data,POSIXtime,jd)}
\end{Highlighting}
\end{Shaded}

\hypertarget{plot-the-systolic-readings-and-show-the-time-limits-of-the-test}{%
\subsection{Plot the Systolic readings and show the time-limits of the
test}\label{plot-the-systolic-readings-and-show-the-time-limits-of-the-test}}

\begin{Shaded}
\begin{Highlighting}[]
\KeywordTok{plot}\NormalTok{(local_df}\OperatorTok{$}\NormalTok{jd,local_df}\OperatorTok{$}\NormalTok{Sys,}\DataTypeTok{pch=}\DecValTok{19}\NormalTok{,}\DataTypeTok{xlab=}\StringTok{"Days"}\NormalTok{,}\DataTypeTok{ylab=}\StringTok{"Sys [mm Hg]"}\NormalTok{,}\DataTypeTok{main=}\StringTok{"Testing effect of 2 vs 3 pills"}\NormalTok{)}
\NormalTok{lim1 <-}\StringTok{ }\FloatTok{18055.0}
\NormalTok{lim2 <-}\StringTok{ }\FloatTok{18075.0}
\KeywordTok{abline}\NormalTok{(}\DataTypeTok{v=}\NormalTok{lim1,}\DataTypeTok{col=}\StringTok{"red"}\NormalTok{)}
\KeywordTok{abline}\NormalTok{(}\DataTypeTok{v=}\NormalTok{lim2,}\DataTypeTok{col=}\StringTok{"red"}\NormalTok{)}
\end{Highlighting}
\end{Shaded}

\includegraphics{bp_tester_1_files/figure-latex/unnamed-chunk-3-1.pdf}

\begin{Shaded}
\begin{Highlighting}[]
\KeywordTok{plot}\NormalTok{(local_df}\OperatorTok{$}\NormalTok{jd,local_df}\OperatorTok{$}\NormalTok{Dia,}\DataTypeTok{pch=}\DecValTok{19}\NormalTok{,}\DataTypeTok{xlab=}\StringTok{"Days"}\NormalTok{,}\DataTypeTok{ylab=}\StringTok{"Dia [mm Hg]"}\NormalTok{,}\DataTypeTok{main=}\StringTok{"Testing effect of 2 vs 3 pills"}\NormalTok{)}
\NormalTok{lim1 <-}\StringTok{ }\FloatTok{18055.0}
\NormalTok{lim2 <-}\StringTok{ }\FloatTok{18075.0}
\KeywordTok{abline}\NormalTok{(}\DataTypeTok{v=}\NormalTok{lim1,}\DataTypeTok{col=}\StringTok{"red"}\NormalTok{)}
\KeywordTok{abline}\NormalTok{(}\DataTypeTok{v=}\NormalTok{lim2,}\DataTypeTok{col=}\StringTok{"red"}\NormalTok{)}
\end{Highlighting}
\end{Shaded}

\includegraphics{bp_tester_1_files/figure-latex/unnamed-chunk-3-2.pdf}

\begin{Shaded}
\begin{Highlighting}[]
\KeywordTok{plot}\NormalTok{(local_df}\OperatorTok{$}\NormalTok{jd,local_df}\OperatorTok{$}\NormalTok{Pulse,}\DataTypeTok{pch=}\DecValTok{19}\NormalTok{,}\DataTypeTok{xlab=}\StringTok{"Days"}\NormalTok{,}\DataTypeTok{ylab=}\StringTok{"Pulse [bpm]"}\NormalTok{,}\DataTypeTok{main=}\StringTok{"Testing effect of 2 vs 3 pills"}\NormalTok{)}
\NormalTok{lim1 <-}\StringTok{ }\FloatTok{18055.0}
\NormalTok{lim2 <-}\StringTok{ }\FloatTok{18075.0}
\KeywordTok{abline}\NormalTok{(}\DataTypeTok{v=}\NormalTok{lim1,}\DataTypeTok{col=}\StringTok{"red"}\NormalTok{)}
\KeywordTok{abline}\NormalTok{(}\DataTypeTok{v=}\NormalTok{lim2,}\DataTypeTok{col=}\StringTok{"red"}\NormalTok{)}
\end{Highlighting}
\end{Shaded}

\includegraphics{bp_tester_1_files/figure-latex/unnamed-chunk-3-3.pdf}

\hypertarget{define-the-two-samples-to-test-with-ks2}{%
\subsection{Define the two samples to test with
KS2}\label{define-the-two-samples-to-test-with-ks2}}

\begin{Shaded}
\begin{Highlighting}[]
\CommentTok{# Define sample with all medicine}
\NormalTok{idx <-}\StringTok{ }\KeywordTok{which}\NormalTok{(local_df}\OperatorTok{$}\NormalTok{jd }\OperatorTok{<}\StringTok{ }\NormalTok{lim1 }\OperatorTok{|}\StringTok{ }\NormalTok{local_df}\OperatorTok{$}\NormalTok{jd }\OperatorTok{>}\StringTok{ }\NormalTok{lim2)}
\CommentTok{# Define sampel with only towmedicines}
\NormalTok{jdx <-}\StringTok{ }\KeywordTok{which}\NormalTok{(local_df}\OperatorTok{$}\NormalTok{jd }\OperatorTok{>}\StringTok{ }\NormalTok{lim1 }\OperatorTok{&}\StringTok{ }\NormalTok{local_df}\OperatorTok{$}\NormalTok{jd }\OperatorTok{<}\StringTok{ }\NormalTok{lim2)}

\CommentTok{# Test on Sys}
\NormalTok{var1totest <-}\StringTok{ }\NormalTok{local_df}\OperatorTok{$}\NormalTok{Sys[idx]}
\NormalTok{var2totest <-}\StringTok{ }\NormalTok{local_df}\OperatorTok{$}\NormalTok{Sys[jdx]}
\NormalTok{ks2 <-}\StringTok{ }\KeywordTok{ks2Test}\NormalTok{(var1totest,var2totest)}
\end{Highlighting}
\end{Shaded}

\begin{verbatim}
## Warning in ks.test(x = x, y = y, alternative = "two.sided"): cannot compute
## exact p-value with ties
\end{verbatim}

\begin{verbatim}
## Warning in ks.test(x = x, y = y, exact = TRUE, alternative = "two.sided"):
## cannot compute exact p-value with ties
\end{verbatim}

\begin{verbatim}
## Warning in ks.test(x = x, y = y, alternative = "less"): cannot compute
## exact p-value with ties
\end{verbatim}

\begin{verbatim}
## Warning in ks.test(x = x, y = y, alternative = "greater"): cannot compute
## exact p-value with ties
\end{verbatim}

\begin{Shaded}
\begin{Highlighting}[]
\NormalTok{ks2}
\end{Highlighting}
\end{Shaded}

\begin{verbatim}
## 
## Title:
##  Kolmogorov-Smirnov Two Sample Test
## 
## Test Results:
##   STATISTIC:
##     D | Two Sided: 0.3801
##        D^- | Less: 0
##     D^+ | Greater: 0.3801
##   P VALUE:
##     Alternative       Two-Sided: 0.0001136 
##     Alternative Exact Two-Sided: 0.0001136 
##     Alternative            Less: 1 
##     Alternative         Greater: 5.678e-05 
## 
## Description:
##  Fri Jul 12 08:50:12 2019
\end{verbatim}

\begin{Shaded}
\begin{Highlighting}[]
\CommentTok{# Test on Dia}
\NormalTok{var1totest <-}\StringTok{ }\NormalTok{local_df}\OperatorTok{$}\NormalTok{Dia[idx]}
\NormalTok{var2totest <-}\StringTok{ }\NormalTok{local_df}\OperatorTok{$}\NormalTok{Dia[jdx]}
\NormalTok{ks2 <-}\StringTok{ }\KeywordTok{ks2Test}\NormalTok{(var1totest,var2totest)}
\end{Highlighting}
\end{Shaded}

\begin{verbatim}
## Warning in ks.test(x = x, y = y, alternative = "two.sided"): cannot compute
## exact p-value with ties
\end{verbatim}

\begin{verbatim}
## Warning in ks.test(x = x, y = y, exact = TRUE, alternative = "two.sided"):
## cannot compute exact p-value with ties
\end{verbatim}

\begin{verbatim}
## Warning in ks.test(x = x, y = y, alternative = "less"): cannot compute
## exact p-value with ties
\end{verbatim}

\begin{verbatim}
## Warning in ks.test(x = x, y = y, alternative = "greater"): cannot compute
## exact p-value with ties
\end{verbatim}

\begin{Shaded}
\begin{Highlighting}[]
\NormalTok{ks2}
\end{Highlighting}
\end{Shaded}

\begin{verbatim}
## 
## Title:
##  Kolmogorov-Smirnov Two Sample Test
## 
## Test Results:
##   STATISTIC:
##     D | Two Sided: 0.3328
##        D^- | Less: 0.0099
##     D^+ | Greater: 0.3328
##   P VALUE:
##     Alternative       Two-Sided: 0.001109 
##     Alternative Exact Two-Sided: 0.001109 
##     Alternative            Less: 0.9934 
##     Alternative         Greater: 0.0005546 
## 
## Description:
##  Fri Jul 12 08:50:12 2019
\end{verbatim}

\begin{Shaded}
\begin{Highlighting}[]
\CommentTok{# Test on Pulse}
\NormalTok{var1totest <-}\StringTok{ }\NormalTok{local_df}\OperatorTok{$}\NormalTok{Pulse[idx]}
\NormalTok{var2totest <-}\StringTok{ }\NormalTok{local_df}\OperatorTok{$}\NormalTok{Pulse[jdx]}
\NormalTok{ks2 <-}\StringTok{ }\KeywordTok{ks2Test}\NormalTok{(var1totest,var2totest)}
\end{Highlighting}
\end{Shaded}

\begin{verbatim}
## Warning in ks.test(x = x, y = y, alternative = "two.sided"): cannot compute
## exact p-value with ties
\end{verbatim}

\begin{verbatim}
## Warning in ks.test(x = x, y = y, exact = TRUE, alternative = "two.sided"):
## cannot compute exact p-value with ties
\end{verbatim}

\begin{verbatim}
## Warning in ks.test(x = x, y = y, alternative = "less"): cannot compute
## exact p-value with ties
\end{verbatim}

\begin{verbatim}
## Warning in ks.test(x = x, y = y, alternative = "greater"): cannot compute
## exact p-value with ties
\end{verbatim}

\begin{Shaded}
\begin{Highlighting}[]
\NormalTok{ks2}
\end{Highlighting}
\end{Shaded}

\begin{verbatim}
## 
## Title:
##  Kolmogorov-Smirnov Two Sample Test
## 
## Test Results:
##   STATISTIC:
##     D | Two Sided: 0.4914
##        D^- | Less: 0.4914
##     D^+ | Greater: 0
##   P VALUE:
##     Alternative       Two-Sided: 1.601e-07 
##     Alternative Exact Two-Sided: 1.601e-07 
##     Alternative            Less: 8.003e-08 
##     Alternative         Greater: 1 
## 
## Description:
##  Fri Jul 12 08:50:12 2019
\end{verbatim}

\hypertarget{example-so-that-we-can-understand-the-d-and-p-value-above}{%
\subsection{Example so that we can understand the D and p value
above}\label{example-so-that-we-can-understand-the-d-and-p-value-above}}

\begin{Shaded}
\begin{Highlighting}[]
\CommentTok{## rnorm - }
   \CommentTok{# Generate Series drawn from different populations:}
\NormalTok{   x =}\StringTok{ }\KeywordTok{rnorm}\NormalTok{(}\DecValTok{34}\NormalTok{)}
\NormalTok{   y =}\StringTok{ }\KeywordTok{rnorm}\NormalTok{(}\DecValTok{76}\NormalTok{)}\OperatorTok{+}\FloatTok{1.2}
  
\CommentTok{## ks2Test - }
   \KeywordTok{ks2Test}\NormalTok{(x, y)}
\end{Highlighting}
\end{Shaded}

\begin{verbatim}
## 
## Title:
##  Kolmogorov-Smirnov Two Sample Test
## 
## Test Results:
##   STATISTIC:
##     D | Two Sided: 0.5015
##        D^- | Less: 0
##     D^+ | Greater: 0.5015
##   P VALUE:
##     Alternative       Two-Sided: 6.285e-06 
##     Alternative Exact Two-Sided: 6.285e-06 
##     Alternative            Less: 1 
##     Alternative         Greater: 7.369e-06 
## 
## Description:
##  Fri Jul 12 08:50:12 2019
\end{verbatim}


\end{document}
